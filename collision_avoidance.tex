\documentclass[journal]{IEEEtran}
\usepackage[english]{babel}

\begin{document}
% can use linebreaks \\ within to get better formatting as desired
\title{Collision avoidance for robot swarms focused on maneuvering and data collection rate from a RTOS approach}

\author{Yu~Shan~Hsieh,~\IEEEmembership{Engineer,~Hewlett Packard,}
   Ignacio~Garita,~\IEEEmembership{Engineer,~Hewlett Packard,}
   and Horacio~Hidalgo,~\IEEEmembership{Engineer,~Hewlett Packard}%
\thanks{}%
% The paper headers
\markboth{Journal of ITCR, April~2013}%
}

% make the title area
\maketitle

% in the abstract or keywords.
\begin{abstract}

\end{abstract}

\section{Introduction}
\IEEEPARstart{I}{n} 

%Acknowledgements
\section*{Acknowledgment}

%Reference
\ifCLASSOPTIONcaptionsoff
  \newpage
\fi
\begin{thebibliography}{1}

\bibitem{Swarm Robotics:Wikipedia}

\emph{Swarm Robotics} 27 Feb 2013, 01:47 UTC. In Wikipedia: The Free Encyclopedia. Wikimedia Foundation Inc.
Encyclopedia on-line.
Available from \textit{http://en.wikipedia.org/wiki/Swarm\_robotics}. Internet. Retrieved 3 March 2013.

\bibitem{How Google's Self-Driving Car Works:IEEE Spectrum}
\emph{How Google's Self-Driving Car Works - IEEE Spectrum} \textit{http://spectrum.ieee.org/automaton/robotics/artificial-intelligence/how-google-self-driving-car-works}.
Spectrum.ieee.org. Retrieved March 3, 2013. 

\bibitem{Swarm Behaviour:Wikipedia}

\emph{Swarm Behaviour} 27 Feb 2013, 01:47 UTC. In Wikipedia: The Free Encyclopedia. Wikimedia Foundation Inc.
Encyclopedia on-line.
Available from \textit{http://http://en.wikipedia.org/wiki/Swarm\_behaviour}. Internet. Retrieved 3 March 2013.

\bibitem{The Kilobot Project:Harvard}

\emph{Kilobot: A Low Cost Scalable Robot System for Collective Behaviors} Michael~Rubenstein, Christian~Ahler, Radhika~Nagpal
Available from \textit{http://www.eecs.harvard.edu/ssr/projects/progSA/kilobot.html}. Internet. Retrieved 3 March 2013.


\end{thebibliography}
\end{document}


