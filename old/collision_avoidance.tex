\documentclass[journal]{IEEEtran}
\usepackage[english]{babel}

\begin{document}
% can use linebreaks \\ within to get better formatting as desired
\title{Collision avoidance for robot swarms focused on maneuvering and data collection rate from a RTOS approach}

\author{Yu~Shan~Hsieh,~\IEEEmembership{Engineer,~Hewlett Packard,}
	Ignacio~Garita,~\IEEEmembership{Engineer,~Hewlett Packard,}
        Horacio~Hidalgo,~\IEEEmembership{Engineer,~Hewlett Packard,}
        and~Francisco~Tenorio,~\IEEEmembership{Engineer,~Hewlett Packard}%
\thanks{}%
% The paper headers
\markboth{Journal of ITCR, March~2013}%
}

% make the title area
\maketitle

% in the abstract or keywords.
\begin{abstract}
The ability to convey critical information promptly and efficiently between robots
swarms is critical in collision avoidance scenarios. Different approaches to data collection
are important considerations that must be taken in Real Time Systems where different �units� must
effectively coordinate spatial movement to prevent collisions. Through the use of the computer
networking concepts usch as the multicast protocol, our team has devised a new method for
data collection in order to meet the timing constraints of a dynamic system such as this. The multicast
concept allows for information management and establishment of policies related to a set of
subscribers to a particular multicast group, which receives messages from a spatial element who
manages and coordinates information and policies between the subscribed elements. The dynamics
of different multicast groups (of which a single subscriber can be a part of many) creates localized
information hubs and information that need not be distributed across entire networks. The resulting
hubs are flexible, sometimes redundant, localized environments that have better adaptability
to the immediate surroundings. Future studies and application of this networking concept can
provide legitimization of the algorithms presented here; for example, use of robots swarms in an
environment which is not homogeneous and where subscribers will obtain only relevant information
to a specific area of this environment.

\end{abstract}

\section{Introduction}
\IEEEPARstart{I}{n} this paper, we will first start off by discussing concepts and existing technologies regarding computer assisted automatons,
real time systems and their typical applications. We will then investigate the possibility of implementing a hypothetical model of such system
in a real world scenario by putting all the pieces together. Finally we will discuss the advantages and disadvantages of such implementation and the implications
it might have of using a real time scheduler to solve the requirements of this system.

\section{Theoretical Framework}
\subsection{Collision Avoidance}
yadda yadda

\subsection{Data collection}
yadda yadda

\subsection{Maneuvering}
yadda yadda

\section{Implementing the model}
yadda yadda

\section{Conclusions}
yadda yadda

%Acknowledgements
\section*{Acknowledgment}
The authors would like to thank professor Francisco Torres of ITCR for putting his effort and time to revise this paper.

%Reference
\ifCLASSOPTIONcaptionsoff
  \newpage
\fi
\begin{thebibliography}{1}

\bibitem{Swarm Robotics:Wikipedia}

\emph{Swarm Robotics} 27 Feb 2013, 01:47 UTC. In Wikipedia: The Free Encyclopedia. Wikimedia Foundation Inc.
Encyclopedia on-line.
Available from \textit{http://en.wikipedia.org/wiki/Swarm\_robotics}. Internet. Retrieved 3 March 2013.

\bibitem{How Google's Self-Driving Car Works:IEEE Spectrum}
\emph{How Google's Self-Driving Car Works - IEEE Spectrum} \textit{http://spectrum.ieee.org/automaton/robotics/artificial-intelligence/how-google-self-driving-car-works}.
Spectrum.ieee.org. Retrieved March 3, 2013. 

\bibitem{Swarm Behaviour:Wikipedia}

\emph{Swarm Behaviour} 27 Feb 2013, 01:47 UTC. In Wikipedia: The Free Encyclopedia. Wikimedia Foundation Inc.
Encyclopedia on-line.
Available from \textit{http://http://en.wikipedia.org/wiki/Swarm\_behaviour}. Internet. Retrieved 3 March 2013.

\bibitem{The Kilobot Project:Harvard}

\emph{Kilobot: A Low Cost Scalable Robot System for Collective Behaviors} Michael~Rubenstein, Christian~Ahler, Radhika~Nagpal
Available from \textit{http://www.eecs.harvard.edu/ssr/projects/progSA/kilobot.html}. Internet. Retrieved 3 March 2013.


\end{thebibliography}
\end{document}


